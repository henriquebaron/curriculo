\documentclass[11pt,a4paper,sans]{moderncv}        % possible options include font size ('10pt', '11pt' and '12pt'), paper size ('a4paper', 'letterpaper', 'a5paper', 'legalpaper', 'executivepaper' and 'landscape') and font family ('sans' and 'roman')

% moderncv themes
\moderncvstyle{casual}                             % style options are 'casual' (default), 'classic', 'banking', 'oldstyle' and 'fancy'
\moderncvcolor{blue}                               % color options 'black', 'blue' (default), 'burgundy', 'green', 'grey', 'orange', 'purple' and 'red'
%\renewcommand{\familydefault}{\sfdefault}         % to set the default font; use '\sfdefault' for the default sans serif font, '\rmdefault' for the default roman one, or any tex font name
%\nopagenumbers{}                                  % uncomment to suppress automatic page numbering for CVs longer than one page

% character encoding
%\usepackage[utf8]{inputenc}                       % if you are not using xelatex ou lualatex, replace by the encoding you are using

\usepackage[english]{babel}
% adjust the page margins
\usepackage[scale=0.75]{geometry}
%\setlength{\hintscolumnwidth}{3cm}                % if you want to change the width of the column with the dates
%\setlength{\makecvtitlenamewidth}{10cm}           % for the 'classic' style, if you want to force the width allocated to your name and avoid line breaks. be careful though, the length is normally calculated to avoid any overlap with your personal info; use this at your own typographical risks...

% personal data
\name{Henrique}{Baron}
\address{Rua Plácido de Castro, 985}{95084-370}{Brasil}% optional, remove / comment the line if not wanted; the "postcode city" and "country" arguments can be omitted or provided empty
\phone[mobile]{+55~(54)~99911-0881}                   % optional, remove / comment the line if not wanted; the optional "type" of the phone can be "mobile" (default), "fixed" or "fax"
\email{henrique.baron@gmail.com}                               % optional, remove / comment the line if not wanted
\social[linkedin]{henriquebaron}                        % optional, remove / comment the line if not wanted
\social[github]{henriquebaron}                              % optional, remove / comment the line if not wanted
\photo[64pt][0.4pt]{picture}                       % optional, remove / comment the line if not wanted; '64pt' is the height the picture must be resized to, 0.4pt is the thickness of the frame around it (put it to 0pt for no frame) and 'picture' is the name of the picture file

% Dados gerais
\newcommand{\localname}{Caxias do Sul -- RS}
% Conteúdo da seção de educação
\newcommand{\educacaoname}{Educação}
\newcommand{\brazilname}{}
\newcommand{\euaname}{~(Estados Unidos)}
\newcommand{\alemanhaname}{~(Alemanha)}
\newcommand{\programacertname}{Programa de Certificação Profissional}
\newcommand{\participacaoname}{Participação nos cursos}
\newcommand{\bachelorengname}{Bacharelado em Engenharia Mecânica}
\newcommand{\intercambioname}{Semestre de intercâmbio}
\newcommand{\tecnicoautomacaoname}{Técnico em Automação Industrial}
% Conteúdo da seção de experiência profissional
\newcommand{\experiencianame}{Experiência profissional}
\newcommand{\atualname}{atual}
% Técnico em P&D - Auttom
\newcommand{\tecnicopedname}{Técnico em Pesquisa e Desenvolvimento}
\newcommand{\tecnicopeddescricao}{Desenvolvimento de softwares para sistemas didáticos e de automação industrial}
\newcommand{\desenvlinguagensname}{Desenvolvimento de aplicações nas linguagens C$\sharp$, C++ e LabVIEW}
\newcommand{\programacaokukaname}{Programação de robôs KUKA em linguagem KRL}
\newcommand{\gestaosoftwaresname}{Gestão e planejamento de projetos de software}
\newcommand{\suportename}{Suporte a clientes}
% Estagiário - Festo
\newcommand{\estagiariofestoname}{Estagiário em desenvolvimento}
\newcommand{\estagiariofestodescricao}{Criação de programas e projetos mecânicos para sistemas didáticos em mecatrônica}
\newcommand{\programacaoclpfestoname}{Programação de CLPs e IHMs Siemens, Allen Bradley e Festo}
\newcommand{\programacaomitsubishiname}{Programação de robôs Mitsubishi em linguagem MELFA V}
\newcommand{\projetosmecanicoscreo}{Projeto de componentes mecânicos no software Creo Parametric}
% Técnico em automação - Auttom
\newcommand{\tecnicoautname}{Técnico em Automação}
\newcommand{\tecnicoautdescricao}{Programação e implantação de sistemas de automação industrial}
\newcommand{\programacaoclpauttomname}{Programação de CLPs, IHMs e controladores de segurança Schneider Electric, Siemens e Allen Bradley}
\newcommand{\testeemcamponame}{Teste e acompanhamento da solução em campo}
% Certificações
\newcommand{\certifname}{Certificações}
\newcommand{\ourowscname}{Medalha de ouro na modalidade Mecatrônica do WorldSkills Competition 2013, realizado em Leipzig (Alemanha)}
\newcommand{\ouroocname}{Medalha de ouro na modalidade Mecatrônica da Olimpíada Nacional do Conhecimento 2012, realizada em São Paulo pelo SENAI}
\newcommand{\estrelaprisioneiraname}{Prêmio Estrela Prisioneira, condedido à nota mais alta entre todas as modalidades na Olimpíada do Conhecimento 2012, realizada em São Paulo pelo SENAI}
% Cursos
\newcommand{\cursosname}{Cursos}
\newcommand{\csavancadoname}{C$\sharp$ Avançado}
\newcommand{\cswinformsname}{C$\sharp$ Windows Forms}
\newcommand{\cswpfname}{C$\sharp$ Windows Presentation Foundation}
\newcommand{\cppbasiconame}{C++ Básico}
\newcommand{\sqldevname}{SQL Desenvolvedor}
\newcommand{\logicaoopname}{Lógica Orientada a Objetos}
\newcommand{\lvcorename}{LabVIEW Core 3}
\newcommand{\lvoopname}{LabVIEW Oriented Object Programming}
\newcommand{\siemenstiaproname}{Siemens STEP7 -- Instruções Avançadas}
\newcommand{\krcprogname}{Programação KR C4 Nível 1}
\newcommand{\excelavname}{Excel Avançado -- Gerencial}
% Idiomas
\newcommand{\idiomasname}{Idiomas}
\newcommand{\portuguesname}{Português}
\newcommand{\inglesname}{Inglês}
\newcommand{\alemaoname}{Alemão}
\newcommand{\linguamaename}{Língua materna}
\newcommand{\fluentename}{Fluente}
% Outras habilidades
\newcommand{\outrosname}{Outras competências}
\newcommand{\sistemassccname}{Sistemas SCC}
\newcommand{\modelagemname}{Modelagem}
\newcommand{\marcacaoname}{Formatação}
\newcommand{\jogosname}{Jogos}
% Projetos
\newcommand{\projetosname}{Projetos}
\newcommand{\ucsmonographdescricao}{Classe \LaTeX\ para formatação de trabalhos conforme o padrão da Universidade de Caxias do Sul. Disponível em}
%
% Textos em inglês
\addto\captionsenglish{
	% Dados gerais
	\renewcommand{\localname}{Caxias do Sul -- Brazil}
	% Educação
	\renewcommand{\educacaoname}{Education}
	\renewcommand{\brazilname}{~(Brazil)}
	\renewcommand{\euaname}{}
	\renewcommand{\alemanhaname}{~(Germany)}
	\renewcommand{\programacertname}{Professional Certificate Program}
	\renewcommand{\participacaoname}{Participation in the courses}
	\renewcommand{\bachelorengname}{Bachelor in Mechanical Engineering}
	\renewcommand{\intercambioname}{Exchange semester}
	\renewcommand{\tecnicoautomacaoname}{Industrial Automation Technician}
	% Experiência profissional
	\renewcommand{\experiencianame}{Professional experience}
	\renewcommand{\atualname}{current}
	% Técnico em P&D - Auttom
	\renewcommand{\tecnicopedname}{Research and Development Technician}
	\renewcommand{\tecnicopeddescricao}{Software development for educational and industry automation systems}
	\renewcommand{\desenvlinguagensname}{Application development in the languages C$\sharp$, C++ and LabVIEW}
	\renewcommand{\programacaokukaname}{Programming of KUKA robots in KRL language}
	\renewcommand{\gestaosoftwaresname}{Planning and management of software projects}
	\renewcommand{\suportename}{Customer support}
	% Estagiário - Festo
	\renewcommand{\estagiariofestoname}{Intern in development}
	\renewcommand{\estagiariofestodescricao}{Creation of programs and mechanical projects for didactic systems for mechatronics}
	\renewcommand{\programacaoclpfestoname}{Programming of Siemens, Allen Bradley and Festo PLCs}
	\renewcommand{\programacaomitsubishiname}{Programming of Mitsubishi robots in MELFA language}
	\renewcommand{\projetosmecanicoscreo}{Mechanical components design in Creo Parametric CAD software}
	% Tecnico em automação - Auttom
	\renewcommand{\tecnicoautname}{Automation Technician}
	\renewcommand{\tecnicoautdescricao}{Programming and deploying of industry automation systems}
	\renewcommand{\programacaoclpauttomname}{Programming of Schneider Electric, Siemens and Allen Bradley PLCs, HMIs, and safety controllers}
	\renewcommand{\testeemcamponame}{On-site testing of the solution}
	% Certificações
	\renewcommand{\certifname}{Certifications}
	\renewcommand{\ourowscname}{Gold medal in Mechatronics at the WorldSkills Competition 2013, held in Leipzig (Germany)}
	\renewcommand{\ouroocname}{Gold medal in Mechatronics at the National Skills Competition 2012, held in São Paulo (Brazil)}
	\renewcommand{\estrelaprisioneiraname}{''Estrela Prisioneira'' Prize, given to the highest mark between all trades of the National Skills Competition 2012, held in São Paulo (Brazil)}
	% Cursos
	\renewcommand{\cursosname}{Courses}
	\renewcommand{\csavancadoname}{C$\sharp$ -- Advanced}
	\renewcommand{\cppbasiconame}{C++ -- Basic}
	\renewcommand{\sqldevname}{SQL for Developers}
	\renewcommand{\logicaoopname}{Object Oriented Logic}
	\renewcommand{\siemenstiaproname}{Siemens STEP7 -- Advanced Instructions}
	\renewcommand{\krcprogname}{KR C4 Programming -- Level 1}
	\renewcommand{\excelavname}{Advanced Microsoft Excel}
	% Idiomas
	\renewcommand{\idiomasname}{Languages}
	\renewcommand{\portuguesname}{Portuguese}
	\renewcommand{\inglesname}{English}
	\renewcommand{\alemaoname}{German}
	\renewcommand{\linguamaename}{Mother language}
	\renewcommand{\fluentename}{Fluent}
	% Outras habilidades
	\renewcommand{\outrosname}{Other competences}
	\renewcommand{\sistemassccname}{SCC systems}
	\renewcommand{\modelagemname}{Modelling}
	\renewcommand{\marcacaoname}{Typesetting}
	\renewcommand{\jogosname}{Games}
	% Projetos
	\renewcommand{\projetosname}{Projects}
	\renewcommand{\ucsmonographdescricao}{\LaTeX\ class for typesetting of documents according to the standards of the University of Caxias do Sul (UCS). Available at}
}
\addto\captionsgerman{
	% Dados gerais
	\renewcommand{\localname}{Caxias do Sul -- Brasilien}
	% Educação
	\renewcommand{\educacaoname}{Bildung}
	\renewcommand{\brazilname}{~(Brasilien)}
	\renewcommand{\euaname}{~(USA)}
	\renewcommand{\alemanhaname}{}
	\renewcommand{\programacertname}{Zertifikatskurs}
	\renewcommand{\participacaoname}{Teilnahme in den Kursen}
	\renewcommand{\bachelorengname}{Bachelor in Maschinenbau}
	\renewcommand{\intercambioname}{Austauschsemester}
	\renewcommand{\tecnicoautomacaoname}{Automatisierungstechniker}
	% Experiência profissional
	\renewcommand{\experiencianame}{Berufserfahrung}
	\renewcommand{\atualname}{aktuell}
	% Técnico em P&D - Auttom
	\renewcommand{\tecnicopedname}{Forschungs- und Entwicklungstechniker}
	\renewcommand{\tecnicopeddescricao}{Softwareentwicklung für Industrieautomatisierung und Lernsysteme}
	\renewcommand{\desenvlinguagensname}{Erstellung von Anwendungen in den Sprachen C$\sharp$, C++ und LabVIEW}
	\renewcommand{\programacaokukaname}{Programmierung von KUKA Robotern in der KRL-Sprache}
	\renewcommand{\gestaosoftwaresname}{Planung und Leitung von Softwareprojekten}
	\renewcommand{\suportename}{Kundensupport}
	% Estagiário - Festo
	\renewcommand{\estagiariofestoname}{Praktikant in Entwicklung}
	\renewcommand{\estagiariofestodescricao}{Programmierungs- und Konstruktionsaufgaben für Lernsysteme im Bereich Mechatronik}
	\renewcommand{\programacaoclpfestoname}{Programmierung von SPS der Marken Siemens, Festo und Allen Bradley}
	\renewcommand{\programacaomitsubishiname}{Programmierung von Mitsubishi Robotern in der MELFA-Sprache}
	\renewcommand{\projetosmecanicoscreo}{Konstruktion im Creo Parametric Software}
	% Técnico em automação - Auttom
	\renewcommand{\tecnicoautname}{Automatisierungstechniker}
	\renewcommand{\tecnicoautdescricao}{Programmierung und Testen von Industrieautomatisierungslösungen}
}


\begin{document}

\makecvtitle

\section{\educacaoname}
\cventry{2019}{\programacertname}{Massachusetts Institute of Technology}{Cambridge--MA\euaname}{}{\participacaoname:
\begin{itemize}
	\item Machine Learning for Big Data and Text Processing: Foundations;
	\item Advanced Machine Learning for Big Data and Text Processing.
\end{itemize}}
\cventry{2011--2019}{\bachelorengname}{Universidade de Caxias do Sul}{Caxias do Sul\brazilname}{}{}
\cventry{2015}{\bachelorengname}{Hochschule Esslingen}{Esslingen\alemanhaname}{}{\intercambioname}

\section{Experience}
\subsection{Vocational}
\cventry{year--year}{Job title}{Employer}{City}{}{General description no longer than 1--2 lines.\newline{}%
Detailed achievements:%
\begin{itemize}%
\item Achievement 1;
\item Achievement 2, with sub-achievements:
  \begin{itemize}%
  \item Sub-achievement (a);
  \item Sub-achievement (b), with sub-sub-achievements (don't do this!);
    \begin{itemize}
    \item Sub-sub-achievement i;
    \item Sub-sub-achievement ii;
    \item Sub-sub-achievement iii;
    \end{itemize}
  \item Sub-achievement (c);
  \end{itemize}
\item Achievement 3.
\end{itemize}}
\cventry{year--year}{Job title}{Employer}{City}{}{Description line 1\newline{}Description line 2}
\subsection{Miscellaneous}
\cventry{year--year}{Job title}{Employer}{City}{}{Description}

\section{Languages}
\cvitemwithcomment{Language 1}{Skill level}{Comment}
\cvitemwithcomment{Language 2}{Skill level}{Comment}
\cvitemwithcomment{Language 3}{Skill level}{Comment}

\section{Computer skills}
\cvdoubleitem{category 1}{XXX, YYY, ZZZ}{category 4}{XXX, YYY, ZZZ}
\cvdoubleitem{category 2}{XXX, YYY, ZZZ}{category 5}{XXX, YYY, ZZZ}
\cvdoubleitem{category 3}{XXX, YYY, ZZZ}{category 6}{XXX, YYY, ZZZ}

\section{Interests}
\cvitem{hobby 1}{Description}
\cvitem{hobby 2}{Description}
\cvitem{hobby 3}{Description}

\section{Extra 1}
\cvlistitem{Item 1}
\cvlistitem{Item 2}
\cvlistitem{Item 3. This item is particularly long and therefore normally spans over several lines. Did you notice the indentation when the line wraps?}

\section{Extra 2}
\cvlistdoubleitem{Item 1}{Item 4}
\cvlistdoubleitem{Item 2}{Item 5\cite{book1}}
\cvlistdoubleitem{Item 3}{Item 6. Like item 3 in the single column list before, this item is particularly long to wrap over several lines.}

\section{References}
\begin{cvcolumns}
  \cvcolumn{Category 1}{\begin{itemize}\item Person 1\item Person 2\item Person 3\end{itemize}}
  \cvcolumn{Category 2}{Amongst others:\begin{itemize}\item Person 1, and\item Person 2\end{itemize}(more upon request)}
  \cvcolumn[0.5]{All the rest \& some more}{\textit{That} person, and \textbf{those} also (all available upon request).}
\end{cvcolumns}

% Publications from a BibTeX file without multibib
%  for numerical labels: \renewcommand{\bibliographyitemlabel}{\@biblabel{\arabic{enumiv}}}% CONSIDER MERGING WITH PREAMBLE PART
%  to redefine the heading string ("Publications"): \renewcommand{\refname}{Articles}
\nocite{*}
\bibliographystyle{plain}
\bibliography{publications}                        % 'publications' is the name of a BibTeX file

% Publications from a BibTeX file using the multibib package
%\section{Publications}
%\nocitebook{book1,book2}
%\bibliographystylebook{plain}
%\bibliographybook{publications}                   % 'publications' is the name of a BibTeX file
%\nocitemisc{misc1,misc2,misc3}
%\bibliographystylemisc{plain}
%\bibliographymisc{publications}                   % 'publications' is the name of a BibTeX file

\clearpage
%-----       letter       ---------------------------------------------------------
% recipient data
\recipient{Company Recruitment team}{Company, Inc.\\123 somestreet\\some city}
\date{January 01, 1984}
\opening{Dear Sir or Madam,}
\closing{Yours faithfully,}
\enclosure[Attached]{curriculum vit\ae{}}          % use an optional argument to use a string other than "Enclosure", or redefine \enclname
\makelettertitle

Lorem ipsum dolor sit amet, consectetur adipiscing elit. Duis ullamcorper neque sit amet lectus facilisis sed luctus nisl iaculis. Vivamus at neque arcu, sed tempor quam. Curabitur pharetra tincidunt tincidunt. Morbi volutpat feugiat mauris, quis tempor neque vehicula volutpat. Duis tristique justo vel massa fermentum accumsan. Mauris ante elit, feugiat vestibulum tempor eget, eleifend ac ipsum. Donec scelerisque lobortis ipsum eu vestibulum. Pellentesque vel massa at felis accumsan rhoncus.

Suspendisse commodo, massa eu congue tincidunt, elit mauris pellentesque orci, cursus tempor odio nisl euismod augue. Aliquam adipiscing nibh ut odio sodales et pulvinar tortor laoreet. Mauris a accumsan ligula. Class aptent taciti sociosqu ad litora torquent per conubia nostra, per inceptos himenaeos. Suspendisse vulputate sem vehicula ipsum varius nec tempus dui dapibus. Phasellus et est urna, ut auctor erat. Sed tincidunt odio id odio aliquam mattis. Donec sapien nulla, feugiat eget adipiscing sit amet, lacinia ut dolor. Phasellus tincidunt, leo a fringilla consectetur, felis diam aliquam urna, vitae aliquet lectus orci nec velit. Vivamus dapibus varius blandit.

Duis sit amet magna ante, at sodales diam. Aenean consectetur porta risus et sagittis. Ut interdum, enim varius pellentesque tincidunt, magna libero sodales tortor, ut fermentum nunc metus a ante. Vivamus odio leo, tincidunt eu luctus ut, sollicitudin sit amet metus. Nunc sed orci lectus. Ut sodales magna sed velit volutpat sit amet pulvinar diam venenatis.

Albert Einstein discovered that $e=mc^2$ in 1905.

\[ e=\lim_{n \to \infty} \left(1+\frac{1}{n}\right)^n \]

\makeletterclosing

%\clearpage\end{CJK*}                              % if you are typesetting your resume in Chinese using CJK; the \clearpage is required for fancyhdr to work correctly with CJK, though it kills the page numbering by making \lastpage undefined
\end{document}


%% end of file `template.tex'.
